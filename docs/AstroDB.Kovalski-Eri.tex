\documentclass[12pt]{article}
\usepackage[utf8]{inputenc}
\usepackage[ukrainian]{babel}
\usepackage{graphicx,float,epsfig,rotating,euscript,amsmath,textcomp,amsfonts,amssymb,color,multirow}
\usepackage[usenames,dvipsnames]{xcolor}
\usepackage{hyperref}
\textheight 257mm \textwidth 17.5cm \hoffset= 0mm \voffset= 0cm
\topmargin -20mm \oddsidemargin -5mm \evensidemargin -5mm

\begin{document}
\fontsize{14}{14}\selectfont
\newpage

{\center \LARGE {\bf Лабораторна робота №1} \\}
{\center \large {\bf Знаходження апексу Сонця методом Ковальського-Ері} \\}

\begin{center}\
  Коваль Анатолій Володимирович \\
\end{center}

\section*{Теоретична частина}
  \begin{equation}
    \begin{cases}
      x &= r\cos\alpha\cos\delta \\
      y &= r\sin\alpha\cos\delta \\
      z &= r\sin\delta \\
    \end{cases}
  \end{equation}

  \begin{equation}
    \begin{cases}
      \tg\alpha &= \dfrac{x}{y} \\
      \tg\delta &= \dfrac{z}{\sqrt{x^2+y^2}} \\
    \end{cases}
  \end{equation}

  \begin{equation}
    \begin{cases}
      \mu_{\alpha}
      = \dfrac{d\alpha}{dt}
      &= \dfrac{\cos\alpha}{r\cos\delta}\dot{y} - \dfrac{\sin\alpha}{r\cos\delta}\dot{x} \\
      \mu_{\delta}
      = \dfrac{d\delta}{dt}
      &= \dfrac{\cos\delta}{r}\dot{z}
      - \dfrac{\sin\delta\cos\alpha}{r}\dot{x}
      - \dfrac{\sin\delta\sin\alpha}{r}\dot{y}
    \end{cases}
  \end{equation}

  В середньому $\Sigma\dot{x_i}=\Sigma\dot{y_i}=\Sigma\dot{z_i}=0$, для окремих зір $\dot{x_i}=\dot{y_i}=\dot{z_i}=0$, де $\mu_{\alpha} = \dfrac{d\alpha}{dt}$ та $\mu_{\delta} = \dfrac{d\delta}{dt}$.

  \begin{equation}
    \begin{cases}
      \dot{x} &= \dot{x_1} - X\\
      \dot{y} &= \dot{y_1} - Y\\
      \dot{z} &= \dot{z_1} - Z\\
    \end{cases},
  \end{equation}

  Де $(\dot{x_1}, \dot{y_1}, \dot{z_1})$ - перекулярний рух зорі, $(-X, -Y, -Z)$ - паралактичне зміщення від руху Сонця. Тепер можемо записати:

  \begin{equation}
    \begin{cases}
      \mu_{\alpha}\cos\delta &= \dfrac{\sin\alpha}{r}X - \dfrac{\cos\delta}{r}Y\\
      \mu_{\delta} &= \dfrac{\sin\delta\cos\alpha}{r}X + \dfrac{\sin\alpha\sin\delta}{r}Y - \dfrac{\cos\delta}{r}Z\\
    \end{cases}
  \end{equation}

  Прийнявши більш пізні поправки:

  \begin{equation}
    \begin{cases}
      kr\mu_{\alpha}\cos\delta &= \sin\alpha X - \cos\delta Y\\
      kr\mu_{\delta} &= \sin\delta\cos\alpha X + \sin\alpha\sin\delta Y - \cos\delta Z\\
    \end{cases}
  \end{equation}

  Знайдені $(X, Y, Z)$ методом найменьших квадратів перетворемо їх з метою отримання $(\alpha, \delta)$:
  \begin{equation}
    \begin{cases}
      \tg\alpha &= \dfrac{Y}{X}\\
      \tg\delta &= \dfrac{Z}{\sqrt{X^2+Y^2}}\\
    \end{cases}
  \end{equation}

\section*{Практична частина}

  Для реалізації розрахунків дані було завантажено з \href{http://cdsarc.u-strasbg.fr/viz-bin/Cat?target=http&cat=I\%2F146&}{I/146} - спостереження північної півкулі каталогу PPM та \href{http://cdsarc.u-strasbg.fr/viz-bin/Cat?target=http&cat=I\%2F193&}{I/193} - спостереження південної півкулі каталогу PPM.

  Користуючись \texttt{python} та \texttt{numpy} реалізовано скрипти обробки та оброблено $378910$ об'єктів. За результатами розрахунків апекс Сонця знаходиться \texttt{(RA) $18^{\texttt{h}}$ $40^{\texttt{m}}$ $23.2 \pm 55.5^{\texttt{s}}$ (dec) $70.17 \pm 0.23^{\circ}$ N}. За сучасними розрахунками координати сонячного апексу \texttt{(RA) $18^{\texttt{h}}$ $03^{\texttt{m}}$ $50.2^{\texttt{s}}$ (dec) $30.00^{\circ}$ N}.

\section*{Висновки}

  Не зважаючи на малість похибки отриманий результат не збігається із вказаним вище \texttt{(RA) $18^{\texttt{h}}$ $03^{\texttt{m}}$ $50.2^{\texttt{s}}$ (dec) $30.00^{\circ}$ N}. Можливо додаткова фільтрація даних: відкидання записів із флагом \texttt{P}, відкидання швидких зір. Та наявність точного значення відстаней до зірок.

\end{document}
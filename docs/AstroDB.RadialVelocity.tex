\documentclass[12pt]{article}
\usepackage[utf8]{inputenc}
\usepackage[ukrainian]{babel}
\usepackage{graphicx,float,epsfig,rotating,euscript,amsmath,textcomp,amsfonts,amssymb,color,multirow}
\usepackage[usenames,dvipsnames]{xcolor}
\usepackage{hyperref}
\textheight 257mm \textwidth 17.5cm \hoffset= 0mm \voffset= 0cm
\topmargin -20mm \oddsidemargin -5mm \evensidemargin -5mm

\begin{document}
\fontsize{14}{14}\selectfont
\newpage

{\center \LARGE {\bf Лабораторна робота №2} \\}
{\center \large {\bf Знаходження апексу Сонця за променевими швидкостями зір} \\}

\begin{center}\
  Коваль Анатолій Володимирович \\
\end{center}

\section*{Теоретична частина}
  Нехай $V_r^{'}$ - перекулярна швидкість зорі та $\gamma$ - кутова відстань зорі до апексу Сонця, тоді $V_r^{''}=-V_{\odot}\cos\gamma$ - параллактичний компонент швидкості зорі.

  \begin{equation}
    V_r = V_r^{'}-V_{\odot}\cos\gamma \Rightarrow V_r^{'} = V_r+V_{\odot}\cos\gamma
  \end{equation}

  Користуючись:

  \begin{equation}
    \begin{cases}
      X &= V_{\odot}\cos D\cos A \\
      Y &= V_{\odot}\cos D\sin A \\
      Z &= V_{\odot}\sin D \\
    \end{cases}
  \end{equation}

  та

  \begin{equation}
    \begin{cases}
      \cos\gamma &= \sin\delta\sin D + \cos\delta\cos D\cos(\alpha - A) \\
      \sin\gamma\cos\psi &= -\cos\delta\sin D + \sin\delta\cos D\cos(\alpha -A) \\
      \sin\gamma\sin\phi &= \cos D\sin(\alpha - A) \\
    \end{cases}
  \end{equation}

  Отримаємо:

  \begin{equation}
    V_{\odot}\cos\gamma = X\cos\alpha\cos\delta + Y\sin\alpha\cos\delta + Z\sin\delta
  \end{equation}

  Приймаючи, що для більшості зір $V_r^{'}\approx0$:

  \begin{equation}
    -V_r = X\cos\alpha\cos\delta + Y\sin\alpha\cos\delta + Z\sin\delta
  \end{equation}

  Знайдені $(X, Y, Z)$ методом найменьших квадратів перетворемо їх з метою отримання $(\alpha, \delta)$:
  \begin{equation}
    \begin{cases}
      \tg\alpha = \dfrac{Y}{X}\\
      \tg\delta = \dfrac{Z}{\sqrt{X^2+Y^2}}\\
    \end{cases}
  \end{equation}

\section*{Практична частина}

  Для реалізації розрахунків дані було завантажено з \href{http://cdsarc.u-strasbg.fr/viz-bin/Cat?target=http&cat=III\%2F239&}{III/239} - спостереження радіальних швидкостей зір.

  Користуючись \texttt{python} та \texttt{numpy} реалізовано скрипти обробки та оброблено $34553$ об'єктів. За результатами розрахунків апекс Сонця знаходиться \texttt{(RA) $19^{\texttt{h}}$ $01^{\texttt{m}}$ $55.00 \pm 0.00^{\texttt{s}}$ (dec) $35.21 \pm 0.0007^{\circ}$ N}. За сучасними розрахунками координати сонячного апексу \texttt{(RA) $18^{\texttt{h}}$ $03^{\texttt{m}}$ $50.2^{\texttt{s}}$ (dec) $30.00^{\circ}$ N}.

\section*{Висновки}

  Не зважаючи на малість похибки отриманий результат не збігається із вказаним вище \texttt{(RA) $18^{\texttt{h}}$ $03^{\texttt{m}}$ $50.2^{\texttt{s}}$ (dec) $30.00^{\circ}$ N}. Проте здобута точність набагато вища порівняно з методом Ковальського-Ері. Можливе покращення точності можна досягнути додатковою фільтрацією зір за малістю перекулярного руху.

\end{document}